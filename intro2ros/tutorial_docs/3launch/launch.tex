\documentclass{article}
\usepackage{graphicx}
\usepackage{geometry}
\usepackage{hyperref}
\usepackage{mathtools}
\usepackage{float}
\usepackage{minted}
\usepackage{xcolor}
\definecolor{LightGray}{rgb}{0.85,0.85,0.85}
\graphicspath{{./}}
\geometry{a4paper, portrait, margin = 1in}
\title{ROS Made Easy \\3: Launch Files and the Parameter Server}
\date{\today}
\author{Aniruddh K Budhgavi \\Enigma, IIIT-B}
\begin{document}
    \maketitle
    \section{Important Note}
    This tutorial was created for \textbf{ROS1 Melodic Morenia}
    on \textbf{Ubuntu 18.04 Bionic Beaver}, in \textbf{June 2020}.
    I expect them to become rapidly out of date. It is my hope
    that Team Enigma will continually maintain and update these tutorials.
    \\
    \\
    This tutorial assumes that you are running Ubuntu, and have at least an
    elementary grasp of Python 2.7 and C/C++ .
    \\
    \\
    All the code for this tutorial is available at \url{https://github.com/aniruddhkb/enigmatutorials}.
    \\
    \\
    The aim of this tutorial is to make you \emph{functional} in ROS, not to make you a master. For 
    that, look elsewhere.
    \section{Launch files}
        \subsection{Introduction}
        \begin{enumerate}
            \item One thing that has been very clunky so far is that we have to manually start 
            the master and every node in its own individual terminal window. This is fine for 
            small environments like Turtlesim, but completely unsuitable for larger projects.
            Further, it is apparent that on many occasions, there are nodes which naturally work
            together and must be spawned together as a group. It is clear that we need a process 
            for launching nodes in bulk, instead of messing around with individual terminals.

            \item All this functionality is provided by \textbf{launch files}.
        \end{enumerate}
        \subsection{A basic launch file}
        \begin{enumerate}
            \item Create a package \texttt{intro2launch} which depends on \texttt{turtlesim}.
            \item Within this package, create a folder called \texttt{launch}.
            \item Within this folder, create \texttt{firstLaunch.launch}:
            \begin{minted}[bgcolor=LightGray]{xml}
    <?xml version="1.0"?>
    <launch>
        <node name="turtlesim" pkg="turtlesim" type="turtlesim_node"/>
        <node name="teleop" pkg="turtlesim" type="turtle_teleop_key"
         launch-prefix="gnome-terminal -e" />
    </launch>
            \end{minted}
    
            To spawn a node, the syntax is \mintinline{xml}{<node name=... pkg=... type=... />}.
            The \texttt{launch-prefix="gnome-terminal --"} option is used to make the node output
            display in a new terminal.
            \item There are a few more options you should familiarize yourself with, including 
            \textbf{required}, \textbf{respawn}, \textbf{ns} and \textbf{output}. You should 
            also take a look at how to pass arguments. All this can be found at \url{http://wiki.ros.org/roslaunch/XML}.
            -- have a look at the tag reference. Also look at the relevant chapter of \emph{A Gentle Introduction to ROS}.
            \item To run this launch file:
            \begin{minted}[bgcolor=LightGray]{bash}
    roslaunch intro2launch firstLaunch.launch
            \end{minted}
            You may notice that this automatically starts a master if no master has been found.
            To shut down all nodes started by this, use Ctrl+C.
        \end{enumerate}
        \subsection{Including other launch files}
        \begin{enumerate}
            \item Now, create \texttt{secondLaunch.launch}.
            \begin{minted}[bgcolor=LightGray]{xml}
    <?xml version="1.0"?>
    <launch>
        <include file="$(find intro2launch)/launch/firstLaunch.launch"/>
    </launch>
            \end{minted}
            Run this in the same way as before. You can also pass args in the include file. Find out 
            how by reading the docs and seeing examples.
        \end{enumerate}
    \section{Looking ahead}
        You can now use launch files to launch multiple nodes at once. This section was especially
        brief because I wish to only give a survey of ROS before spending much more time 
        on simulation and control. 
        \\
        \\
        Now, we are ready to start creating models of our robots for simulation.
\end{document}