\documentclass{article}
\usepackage{graphicx}
\usepackage{geometry}
\usepackage{hyperref}
\usepackage{mathtools}
\usepackage{float}
\usepackage{minted}
\graphicspath{{./}}
\geometry{a4paper, portrait, margin = 1in}
\title{ROS Made Easy//0: Introduction and Installation}
\date{\today}
\author{Aniruddh K Budhgavi \\Enigma, IIIT-B}
\begin{document}
    \maketitle
    \section{Important Note}
    This tutorial was created for \textbf{ROS1 Melodic Morenia}
    on \textbf{Ubuntu 18.04 Bionic Beaver}, in \textbf{June 2020}. These
    tutorials expect that you are running a version of Ubuntu.
    I expect them to become rapidly out of date. It is my hope
    that Team Enigma will continually maintain and update these tutorials.
    \section{Introduction: The motivation behind ROS}
    In any field of software engineering, instead
    of repeatedly reinventing the wheel, it is better to
    reuse code. This is accomplished through packages and libraries.
    This same principle of code reuse is true for robotics.
    \\
    \\
    ROS, or the \textbf{Robot Operating System} provides this
    functionality. ROS allows you to separate the problem of
    low-level hardware control from things like motion planning and
    perception. It provides a standardised and portable mechanism for communication
    between multiple processes -- both within a system and across a network.
    It provides tools for robot visualization and simulation
    and for common tasks like inverse kinematics. ROS can be thought
    of as an operating system because it provides all the functions
    that are expected of one. For more details, see \url{http://wiki.ros.org/ROS/Introduction}
    or the first chapter of \emph{A gentle introduction to ROS}.
    \\
    \\
    Well-designed ROS systems separate the low-level control of
    hardware and the high-level decision-making into 
    two (or more) separate programs.
    \section{Core concepts}
    Dumby
    \section{Installation}
    \begin{enumerate}
        \item Follow the installation instructions at \url{http://wiki.ros.org/melodic/Installation/Ubuntu}
        and do a \textbf{full installation}. This is very important, because it provides 
        many add-ons, including the \textbf{Gazebo simulator} baked into ROS. Getting them 
        later on and integrating them with ROS later on may be difficult.

        \item Ensure that you add the following line to {\texttt ~/.bashrc}:
        \begin{minted}{bash}
            source /opt/ros/setup.bash
        \end{minted}
        \item Close all active terminals and open a fresh terminal.
        \item Run the following command:
        \begin{minted}{bash}
            export | grep ROS
        \end{minted}
        which should display a bunch of environment variables.
    \end{enumerate}
\end{document}