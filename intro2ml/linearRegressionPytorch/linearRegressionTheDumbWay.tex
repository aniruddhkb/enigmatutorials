\documentclass{article}
\usepackage{graphicx}
\usepackage{geometry}
\usepackage{hyperref}
\usepackage{mathtools}
\usepackage{float}
\usepackage{minted}
\graphicspath{{./}}
\geometry{a4paper, portrait, margin = 1in}
\title{2: Linear Regression in Pytorch, the Dumb Way}
\date{\today}
\author{Aniruddh K Budhgavi \\Enigma, IIIT-B}
\begin{document}
    \maketitle 
    \section{Introduction}
    Constructing machine learning models using Numpy is a fantastic way 
    of understanding the nitty-gritties of a particular method, but it is 
    not an ideal way for more complex models or in production environments.
    Instead, we use libraries like Pytorch, Tensorflow and Keras. These libraries
    ensure that you spend less time reinventing the wheel, leaving you free to fine-tune
    the model without getting bogged down by details. They also make your training process
    faster by utilizing the GPU (if you have one).
    \\
    My personal preference is towards Pytorch (anyone who has used Tensorflow 
    1.X will understand why). Pytorch has enough utilities to make the process painless
    but is flexible when it comes to customizations.
    \section{Installation}
    It is worth putting a separate section for this because of a few details. The primary
    complication is due to \textbf{CUDA}, a GPU computing API provided by NVIDIA. The GPU 
    implementation of Pytorch uses CUDA, so \textbf{you can only use Pytorch with a CUDA-
    capable NVIDIA GPU}. You can check if your GPU is supported at \url{https://developer.nvidia.com/cuda-gpus}.
    \\
    You can install Pytorch from \url{https://pytorch.org/get-started/locally/}. There are 
    various package managers and builds which are available.
    \begin{enumerate}
        \item On Windows, I recommend that you use the Anaconda package manager. Use the latest 
            stable version of Pytorch, and be sure to check the CUDA version (later the better).
        \item On Ubuntu, you can install using Pip without any troubles. \textbf{You may have to
        install CUDA manually}.To do so, visit \url{https://developer.nvidia.com/cuda-downloads}.
        \item In case your PC is underpowered, you can always run your models in Kaggle.
        In fact, that is what I do for Deep Learning models, even though I have a GTX 1050. 
        Kaggle provides (as of writing) 30 hrs per week of an NVIDIA Tesla P100 GPU. 
        You can even natively import Kaggle datasets without any trouble.
    \end{enumerate}
    To check if the installation was successful, open a Jupyter notebook and run:
    \begin{minted}{python}
        import torch
    \end{minted}
\end{document}